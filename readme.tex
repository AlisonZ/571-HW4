\documentclass{article}
\usepackage{underscore}

\begin{document}
    \title{Homework 4}
    \author{Alison Zerbe}
    \maketitle{}
   \begin{enumerate}
    \item Structure of the Program \\\\
    For my Oracle I created two classes, Oracle and Token. \\
    Tokens are created as each word is read in from the parsed file. They contain the 
    index, head, word and POS of each token.\\\\
    The Oracle hold the buffer, stack and transitions. A new Oracle instance is created for each parsed grouping.
    The Oracle class also contains the logic related to determining if the terminal case has been reached, and all functions
    required for performing the three transition options of LeftArc, RightArc and Shift \\\\
    hw4_oracle.py drives the program and reads in the input, creates class instances, manages the recursive function to go through the stack and buffer, 
    and prints the results to the specified output file.

    \item Issues \\\\
    This was fairly straightforward. The main issues were realizing that when I shift I have to put back whatever tokens I had popped from the stack.
    I also found it tricky at first to correlate the head and index relationships between the two tokens and left and right arcs.
    The final thing that was a little difficult was how to handle the ROOT. I'm still not entirely happy with how I handled it and think an 
    improvement would be to verify that the last item in the Stack is in fact a ROOT. Also, I defaulted to the ROOT being a RightArc always, but I feel
    this is very English-centric.

    \item Running the Program \\
   \end{enumerate}
\end{document}